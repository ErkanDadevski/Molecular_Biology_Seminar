% !TEX TS-program = pdflatex
% !TEX encoding = UTF-8 Unicode

% This is a simple template for a LaTeX document using the "article" class.
% See "book", "report", "letter" for other types of document.

\documentclass[11pt]{article} % use larger type; default would be 10pt

\usepackage[utf8]{inputenc} % set input encoding (not needed with XeLaTeX)

%%% Examples of Article customizations
% These packages are optional, depending whether you want the features they provide.
% See the LaTeX Companion or other references for full information.

%%% PAGE DIMENSIONS
\usepackage{geometry} % to change the page dimensions
\geometry{a4paper} % or letterpaper (US) or a5paper or....
% \geometry{margin=2in} % for example, change the margins to 2 inches all round
% \geometry{landscape} % set up the page for landscape
%   read geometry.pdf for detailed page layout information

\usepackage{graphicx} % support the \includegraphics command and options

% \usepackage[parfill]{parskip} % Activate to begin paragraphs with an empty line rather than an indent

%%% PACKAGES
\usepackage{booktabs} % for much better looking tables
\usepackage{array} % for better arrays (eg matrices) in maths
\usepackage{paralist} % very flexible & customisable lists (eg. enumerate/itemize, etc.)
\usepackage{verbatim} % adds environment for commenting out blocks of text & for better verbatim
\usepackage{subfig} % make it possible to include more than one captioned figure/table in a single float
\usepackage[dvipsnames]{xcolor}
% These packages are all incorporated in the memoir class to one degree or another...

%%% HEADERS & FOOTERS
\usepackage{fancyhdr} % This should be set AFTER setting up the page geometry
\pagestyle{fancy} % options: empty , plain , fancy
\renewcommand{\headrulewidth}{0pt} % customise the layout...
\lhead{}\chead{}\rhead{}
\lfoot{}\cfoot{\thepage}\rfoot{}

%%% SECTION TITLE APPEARANCE
\usepackage{sectsty}
\allsectionsfont{\sffamily\mdseries\upshape} % (See the fntguide.pdf for font help)
% (This matches ConTeXt defaults)

%%% ToC (table of contents) APPEARANCE
\usepackage[nottoc,notlof,notlot]{tocbibind} % Put the bibliography in the ToC
\usepackage[titles,subfigure]{tocloft} % Alter the style of the Table of Contents
\renewcommand{\cftsecfont}{\rmfamily\mdseries\upshape}
\renewcommand{\cftsecpagefont}{\rmfamily\mdseries\upshape} % No bold!

%%% END Article customizations

%%% The "real" document content comes below...

\title{Transformer-Encoders}
\author{The Author}
%\date{} % Activate to display a given date or no date (if empty),
         % otherwise the current date is printed 

\begin{document}
\maketitle
\tableofcontents
\listoffigures
\listoftables
\section{Abstract}
10pages no ref 
\section{Introduction}
Antimicrobial Resistance of pathogens (AMR) poses a critical and escalating threat to global public health. Recognized by the World Health Organization (WHO) as a major concern, AMR minimizes the effectiveness of antibiotics, leading to far more severe illnesses, increased mortality rate and higher healthcare needs (Hu et al. 2024). The currently used culture-based antimicrobial susceptibility testing (AST) is time-consuming, making the use of a broader spectrum antibiotic more attractive, favouring the emergence of resistent pathogens (Barbosa et al. 2000, \textcolor{blue}{apparently in })\cite{impact00}. This reinforces the need for faster AMR prediction.\newline 
Whole-genome sequencing (WGS) offers an option for this, by providing a quick way to assess genetic information, but it's speed and accuracy is dependent on advanced computational tools.(Su18 )
This \textcolor{blue}{paper?seminar?report?} explores the role of Encoder architectures in the prediction of AMR directly from genomic sequences. As part of the Transformers architectures, Encoders represent a paradigm shift since their introduction in 2017 (Mowbray25). In particular the self-attention characteristic of these Transformer models will be examined in the learning of biological patterns within DNA sequences. This capability is crucial for the identification of resistent mutations and thereby predicting AMR phenotypes.
To achieve this, this report will start with an overview of machine learning models for sequence analysis, highlighting the advancements from traditional models to modern neural networks. Following this, the architectural principles of encoder only Transformer models will be showcased, explaining their key components. Subsequently the practical implementation using a pre-trained encoder model to predict the resistance of cefoxitin in Staphylococcus aureus strands, will be described. Finally, the results of various models will be presented with a discussion of encoder models for AMR prediction. \textcolor{blue}{and possible future works}
\section{neuralnetworks other models}
\subsection{Data Representation}
To apply machine learning to the Straphylococcus aureus gene, biological sequences must be transformed to numerical representations. Traditional approaches work with the One-Hot Encoding approach, which maps nucleotides to binary vectors or k-mer frequency analysis, which counts the occurence of short subsequences (Angermüller). These methods however often result in high-dimensional, sparse data which fails to capture the relationships between nucleotides. Modern Deep Learning approaches address this by utilizing Learned Embeddings, where nucleotides are mapped to dense vectors in a continuous space, allwoing the model to learn mathematical similarities between biologically similar sequences (Zou et all 2019)
\subsection{Deep Learning}
Early computational approaches to AMR prediction relied on supervised algorithms like Support Vector machines (SVM) and Random Forests. These models usually require extensive feature engineering and prior domain knowledge regarding specific resistance markers (Yang 2020)
Deep Learning introduced architectures capable of automatic feature extraction. Convolutional Neural Networks (CNNs) apply sliding filters to sequence data, but struggle to model dependencies between distant parts of sequences, due to limited receptive fields. Recurrent Neural Networks (RNNs) and Long Short-Term Memory (LSTM) models, on the other hand process data sequentially, making the theoretically suitable for analysis of genomic sequences. RNNs often fail to retain context over long sequences due to the vanishing gradient problem (Vaswani et al 2017)
The limitations on these models, especially with long-range dependencies in DNA, necessitated the development of the Transformer architecture.
\section{Transformers}
\textcolor{blue}{heremaybesoimething}
\subsection{Architecture}
The limitations of sequential processing in RNNs were addressed by the introductio of the Transformer architecture. (Vaswani 2017). For the task of AMR prediciton, the Encoder part of this architecture was utilized.\newline
Unlike neural networks, as CNNs, which compare local motifs, or RNNs which compare the previous steps, the Encoder models use a self-attention mechanism, which allows the model to weigh the relevance of every nucleotide in the sequence against every other nucleotide simultaneously. This enables the capture long-range dependencies, which are crucial in genomics, since the resisstent mutation at one site may depend structurally on a distant region within the same genome. (Devlin 2018)(Thomas STructural Bases (noaccess)). \newline Because the encoder processes the entire sequence in parallel, there is no order to follow (Weiss2021). To resolve this, positional encodings are added to the input embeddings to retain information about the relative positions of nulceotides. The model is typically trained using Masked Language Modeling \textcolor{blue}{checkhere}, were a percentage of the input sequence is hidden and the model learns to reconstruct the missing nucleotides based on the contest.\newline
\subsection{Biological application}
Encoders facilitate Transfer Learning. In a biological context, labeled data is scarce. The provided data set contains around 800 training samples. A deep network trained from scracth in a data set of this size, would lead to overfitting of the model.\newline
To overcome this, a pre-trained, and then fine-tuned approach was chosen. Models such as DNABERT (Ji2021) are pretrained on unlabeled bacterial genome databases, learning the fundamental properties of DNA. This pre-trained model is then fine-tuned to specific tasks, in this case the prediction of genomic sequences resistent to Cefoxitin. \newline There are models trained on whole amino acid sequences, called Protein Language Models (PLMs), like ProtBERT, the data provided was in form of raw nucleotide inputs, for which genomic models, such as the chosen DNABERT, are specifically optimized.
\section{Work done}
\subsection{different architectures}
\subsection{measures }
\subsection{code inclusion}
	what was done
\section{Results}
\subsection{various arch result comparisons}
	what we got
	including comparisons
different dropout values 
freeze unfreeze base model
train a few epochs\newline
try to get good results\newline
show difference between different models
imbalance and statistics
\section{Discussion}
meaning of the results
\section{related works}

\section{References}
impact00
 Teresa M. Barbosa, Stuart B. Levy
  Addison Wesley, Massachusetts,
  2nd Edition,
  1994.
\end{document}
